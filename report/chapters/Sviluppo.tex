\documentclass[main.tex]{subfiles}

\usepackage{pythonhighlight}
\usepackage{xcolor}
\usepackage{listings}

\definecolor{mGreen}{rgb}{0,0.6,0}
\definecolor{mGray}{rgb}{0.5,0.5,0.5}
\definecolor{mPurple}{rgb}{0.58,0,0.82}
\definecolor{backgroundColour}{rgb}{0.95,0.95,0.92}

\lstdefinestyle{CStyle}{
    backgroundcolor=\color{backgroundColour},   
    commentstyle=\color{mGreen},
    keywordstyle=\color{magenta},
    numberstyle=\tiny\color{mGray},
    stringstyle=\color{mPurple},
    basicstyle=\footnotesize,
    breakatwhitespace=false,         
    breaklines=true,                 
    captionpos=b,                    
    keepspaces=true,                 
    numbers=left,                    
    numbersep=5pt,                  
    showspaces=false,                
    showstringspaces=false,
    showtabs=false,                  
    tabsize=2,
    language=C
}

\lstdefinestyle{yaml}{
     backgroundcolor=\color{backgroundColour}, 
     basicstyle=\color{blue}\footnotesize,
     rulecolor=\color{black},
     string=[s]{'}{'},
     stringstyle=\color{blue},
     comment=[l]{:},
     numbers=left,
     commentstyle=\color{black},
     morecomment=[l]{-}
 }

\begin{document}
\section{Sviluppo}\label{sec:sviluppo}

\subsection{Binario}
Come visto nella fase di reverse engineering, il codice sorgente non differisce poi molto dal suo equivalente decompilato né differisce dall'analisi già effettuata.\\

main.c
\begin{lstlisting}[style=CStyle]
#include "bof.h"

t_pin	get_pin()
{
    FILE	*urandom;
    t_pin	pin;

    urandom = fopen("/dev/urandom", "r");
    if (urandom == NULL) {
        printf("Failed to open /dev/urandom\n");
        exit(1);
    }
    if (fread(pin.b, sizeof(char), 4, urandom) != 4) {
        printf("Failed to read from /dev/urandom\n");
        fclose(urandom);
        exit(1);
    }
    fclose(urandom);
    return (pin);
}

int	check_pin(u_int32_t pin)
{
    if (pin == get_pin().u)
    {
        puts("Access granted.\n");
        system("/bin/sh");
        return (0);
    }
    puts("Access denied.\n");
    return (1);
}

int	main(void)
{
    char	user[64];
    t_pin	password;

    memset(user, 0, sizeof(user));
    setvbuf(stdin, NULL, _IONBF, 0);
    setvbuf(stdout, NULL, _IONBF, 0);
    setvbuf(stderr, NULL, _IONBF, 0);

    printf("Shoutout at the user of the day: user_%llx\n", main);
    printf("Enter your username: ");
    read(0, user, 0x64);
    printf("Hello %s, what's your pin?\n", user);
    read(0, password.b, 0x64);
    check_pin(password.u);
    return (0);
}
\end{lstlisting}

bof.h
\begin{lstlisting}[style=CStyle]
#ifndef BOF_H
# define BOF_H

# include <stdio.h>
# include <stdlib.h>

typedef union s_pin {
    u_int32_t   u;
    char        b[4];
}	t_pin;

#endif
\end{lstlisting}

\subsection{Container}

Ora serve rendere il binario un vero e propio servizio al quale si può accedere da remoto, qui entra in gioco Docker; per prima cosa creo un Dockerfile, dove poi inzio scegliendo come immagine di partenza un ubuntu:22.04, ci installo socat, che servirà, in questo caso, a rendere il binario un server; dopodiché aggiungo un utente da poi usare dentro il container, copio i file necessari (ovvero, il binario e una flag), espongo una porta da usare e avvio socat nell'entrypoint.

\begin{python}
FROM ubuntu:22.04

RUN apt-get update && \
	apt-get install -y socat

RUN useradd -d /home/chall/ -m -p chall -s /bin/bash chall
RUN echo "chall:chall" | chpasswd

COPY ./remote_login /home/chall/remote
COPY ./flag.txt /home/chall/flag.txt
RUN chmod +x /home/chall/remote

USER chall
WORKDIR /home/chall

EXPOSE 1337

ENTRYPOINT ["socat","TCP-LISTEN:1337,reuseaddr,fork","EXEC:./remote"]
\end{python}

Ora ci manca solo l'ultimo passo per rendere tutto più semplice e comodo da deployare, cioè il compose; qui è riportata la semplice configurazione per build locale, con porta forwardata sulla 1337, e in particolare i limiti sulle risorse e il filesistem readonly per evitare che chi in futuro dovesse accedere a questa challenge di fare danni alla challenge stessa ne alla macchina sulla quale sta operando.

\begin{lstlisting}[style=yaml]
services:
  remote_login:
    build: ./remote
    ports:
      - 1337:1337
    stdin_open: true
    tty: true
    read_only: true
    deploy:
      restart_policy:
        condition: any
        delay: 5s
        window: 10s
      resources:
        limits:
          cpus: '2'
          memory: 1000M
\end{lstlisting}


\end{document}